\subsection{Index types}

We have defined a named index set as a pair $\nset{i}{X}$, where $\name{i}$ is a name and $X$ is a set, usually $[n]$ for some $n$. In this section, we consider some other possibilities for~$X$.

\subsubsection{Non-integral types}

The sets $X$ don't have to contain integers. For example, if $V$ is the vocabulary of a natural language ($V = \{ \text{cat}, \text{dog}, \ldots \}$), we could define a matrix of word embeddings:
\begin{align*}
  E &\in \mathbb{R}^{\nidx{vocab}{V} \times \nidx{emb}{d}}.
\end{align*}

\subsubsection{Integers with units}

If $\name{u}$ is a symbol and $n > 0$, define $[n]\name{u} = \{1\name{u}, 2\name{u}, \ldots, n\name{u}\}$. You could think of $\name{u}$ as analogous to a physical unit like kilograms. The set $[n]\name{u}$ could be used as an index set, which would prevent the index from being aligned with another index that uses different units. For example, if we want to define a tensor representing an image, we might write
\[ A \in \mathbb{R}^{\nidx{height}{[h]\name{px} \times \nidx{width}{[w]\name{px}}}}. \]

\subsubsection{Tuples of integers}

An index set could also be $[m] \times [n]$, which would be a way of sneaking ordered indices into named tensors, useful for matrix operations. For example, instead of defining an $\text{inv}$ operator that takes two subscripts, we could write
\begin{align*}
  A &\in \mathbb{R}^{\nidx{d}{{m\times n}}} = \mathbb{R}^{\nidx{d}{{[m]\times [n]}}} \\
  B &= \nfun{d}{inv} A.
\end{align*}
We could also define an operator $\circ$ for matrix-matrix and matrix-vector multiplication:
\begin{align*}
  c &\in \mathbb{R}^{\nidx{d}{n}} \\
  D &= A \nbin{d}{\circ} B \nbin{d}{\circ} c.
\end{align*}
