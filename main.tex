\pdfoutput=1
\documentclass[10pt]{article}
\usepackage[accepted]{tmlr} 
\let\authorname\name \let\name\undefined % avoid conflict with our \name macro
\usepackage{amsmath}
\usepackage{amssymb}
\usepackage{mathtools}
\usepackage{parskip}
\usepackage{natbib}

%%% The order of the following is important
\usepackage[hidelinks]{hyperref}
\usepackage{amsthm}
\usepackage[capitalize]{cleveref}
\newtheorem{theorem}{Theorem}
\newtheorem{proposition}[theorem]{Proposition}
\theoremstyle{definition}
\newtheorem{definition}{Definition}
\newtheorem{example}[definition]{Example}
%%%

% Make DOIs clickable
\newcommand{\doi}[1]{\href{https://doi.org/#1}{doi: #1}}

\usepackage{namedtensor}

\DeclareMathOperator*{\softmax}{softmax}
\newcommand{\reals}{\mathbb{R}}
\newcommand{\restrict}[2]{\mathopen{}\left.#1\right|_{#2}}
\newcommand{\eqby}[1]{\stackrel{\text{(\ref{#1})}}{=}}
\makeatletter
\newcommand\yestag{\incr@eqnum\tag{\number\value{equation}}}
\makeatother
\allowdisplaybreaks

\DeclareMathOperator{\ind}{ind}
\DeclareMathOperator{\rec}{rec}
\DeclareMathOperator{\shp}{shp}
\newcommand{\nmatrix}[3]{#1\begin{array}[b]{@{}c@{}}#2\\\begin{bmatrix}#3\end{bmatrix}\end{array}}
\newcommand{\inax}{^*} % diacritic for input axis in differentiation
\newcommand{\ddx}[1]{\frac{\partial #1}{\partial X\inax}}

% axis names
\ndef{\ax}{ax}
\ndef{\dd}{d}
\ndef{\layer}{layer}
\ndef{\seq}{seq}
\ndef{\subseq}{subseq}
\ndef{\key}{key} \ndef{\val}{val}
\ndef{\heads}{heads}
\ndef{\batch}{batch}
\ndef{\inp}{input} \ndef{\hidden}{hidden} \ndef{\out}{out}
\ndef{\height}{height} \ndef{\width}{width} \ndef{\chans}{chans}
\ndef{\kernel}{kernel} \ndef{\kh}{kh} \ndef{\kw}{kw}
\ndef{\vocab}{vocab}
\ndef{\classes}{classes}
\ndef{\state}{state}
\ndef{\emb}{emb}

\newcommand{\liftVB}{lift}
\newcommand{\liftingVBG}{lifting}
\newcommand{\liftsVBZ}{lifts}
\newcommand{\liftedVBN}{lifted}
\newcommand{\liftNN}{lift} % result of lifting
\newcommand{\LiftingNN}{Lifting} % act of lifting
\newcommand{\liftingNN}{lifting} % act of lifting

\title{Named Tensor Notation}
\author{\authorname David Chiang \\ \addr University of Notre Dame \AND \authorname Alexander M. Rush \\ \addr Cornell University \AND \authorname Boaz Barak \\ \addr Harvard University}
\def\month{12}
\def\year{2022}
\def\openreview{https://openreview.net/forum?id=hVT7SHlilx}

\begin{document}

\maketitle

\begin{abstract}
We propose a notation for tensors with named axes, which relieves the author, reader, and future implementers of machine learning models from the burden of keeping track of the order of axes and the purpose of each. The notation makes it easy to \liftVB{} operations on low-order tensors to higher order ones, for example, from images to minibatches of images, or from an attention mechanism to multiple attention heads.

After a brief overview and formal definition of the notation, we illustrate it through several examples from modern machine learning, from building blocks like attention and convolution to full models like Transformers and LeNet. We then discuss differential calculus in our notation and compare with some alternative notations. Our proposals build on ideas from many previous papers and software libraries. We hope that our notation will encourage more authors to use named tensors, resulting in clearer papers and more precise implementations.
\end{abstract}

\section{Introduction}
\label{sec:intro}
Formal descriptions of neural networks primarily adopt the notation of vectors and matrices from applied linear algebra~\citep{goodfellow2016deep}. When used to describe vector spaces, this notation is both concise and unambiguous. However, when applied to neural networks, these properties are lost. Consider the equation for attention as notated in the Transformer paper \citep{vaswani+:2017}:
\[ \text{Attention}(Q, K, V) = \left( \softmax \frac{QK^\top}{\sqrt{d_k}} \right) V. \]
The equation relates $Q$, $K$, and $V$ (for query, key, and value, respectively) as sequences of feature vectors, packed into possibly identically-sized matrices. While concise, this equation is ambiguous. Does the product $QK^\top$ sum over the sequence, or over the features? We know that it sums over columns, but there is not enough information to know what the columns represent. Is the softmax taken over the query sequence or the key sequence? The usual notation does not offer an answer. Perniciously, the implementation of an incorrect interpretation might still run without errors. With the addition of more axes, like multiple attention heads or multiple sentences in a minibatch, the notation becomes even more cumbersome. 

We propose an alternative mathematical notation for tensors with \emph{named axes}.\footnote{%
We follow NumPy in using the term \emph{axis}. Other possible terms would be \emph{index}, \emph{dimension}, \emph{way}, or \emph{mode} \citep{tucker:1964}, but we felt that \emph{axis} had the least potential for confusion.} The notation has a formal underpinning, but is hopefully intuitive enough that machine learning researchers can understand it without much effort. 
%
In named tensor notation, the above equation becomes
\begin{align*}
  \text{Attention} \colon \mathbb{R}^{\key} \times \mathbb{R}^{\seq \times \key} \times \mathbb{R}^{\seq \times\val} &\rightarrow \mathbb{R}^{\val} \\
  \text{Attention}(Q,K,V) = \left( \nfun{\seq}{softmax} \frac{Q \ndot{\key} K}{\sqrt{|\key|}} \right) \ndot{\seq} V.
\end{align*}

The type signature introduces three named axes: the $\key$ axis is for features of queries and keys, the $\val$ axis is for features of values, and the $\seq$ axis is for tokens in a sequence. (Please see \cref{sec:goodnames} for an explanation of our naming convention.) This notation makes the types of each input tensor explicit. Tensor $Q$ is a query vector that is compared with key vectors, so it has a $\key$ axis. Tensor $K$ is a sequence of key vectors, so it has $\seq$ and $\key$ axes. Tensor $V$ is a sequence of value vectors, so it has $\seq$ and $\val$ axes. Unlike with matrix notation, the reader is not required to remember whether $\seq$ corresponds to rows or columns in either of these tensors.

The function itself uses the named axes to precisely apply operations.  The expression $Q \ndot{\key} K$ is a dot product over the $\key$ axis shared between $K$ and $Q$; there is no ambiguity about rows or columns. Similarly, the softmax function is annotated with the axis along which it is applied, removing any ambiguity or reliance on convention. 

Furthermore, named tensor notation naturally extends to \textit{\liftingVBG} (also known as vectorizing and/or broadcasting) a function to tensors with more axes. For example, if instead of being a tensor with the single axis $\key$, $Q$  has three axes $\key$, $\seq$ and $\batch$  (corresponding to tokens of a sequence and examples in a minibatch, respectively) then the $\text{Attention}$ function works as written, acting on each example in a minibatch in parallel.
%That is, in this case $\text{Attention}(Q,K,V)$ equals the tensor $A$ with axes $\val$, $\seq$ and $\batch$, such that for every index $s$ of $\seq$ and $b$ of $\batch$, the corresponding element of $A$ is obtained by applying $\text{Attention}$ to the corresponding restriction of $Q$ (together with $K$ and $V$).
Similarly, we can also add a $\heads$ axis to the inputs to get multiple attention heads.
These additional axes are often elided in neural network papers, possibly avoiding notational complexity, but possibly also hiding critical model details. 

\textbf{Our contributions.}  This work proposes a \emph{mathematical notation} for named tensors and a fully specified \emph{semantic interpretation} for the notation.
Through examples, we demonstrate that this notation enables specifying machine learning models and operations in a succinct yet precise manner.
The need for named tensors has been recognized by several software packages, including xarray \citep{xarray}, Nexus \citep{chen2017typesafe}, tsalib \citep{tsalib}, axisarrays~\citep{axisarrays}, NamedTensor \citep{namedtensor}, PyTorch \citep{named-tensors}, Dex~\citep{dex}, JAX \citep{jax_xmap}, einops \citep{einops}, and torchdim \citep{torchdim}. While our notation is inspired by these efforts, our focus is on mathematical notation to be used in papers, whereas previous efforts have focused on code. Our hope is that our notation will be adopted by authors, leading to clearer, more replicable papers, and that this, in turn, will encourage more implementers to adopt named tensor libraries, leading to clearer, more correct code.

%The source code for this document can be found at \url{https://github.com/namedtensor/notation/}. We invite anyone to make comments on this proposal by submitting issues or pull requests on this repository.


\section{Named Tensors}
In standard notation, a vector, matrix, or tensor is indexed by an integer or sequence of integers; if it has dimensions $n_1,\ldots,n_r$, it can be thought of as a map that takes as input $(i_1,\ldots,i_r) \in [n_1]\times \cdots \times [n_r]$ and outputs a real number (or an element of a different field).
For example, if $A \in \reals^{3\times3}$, then the order of the two axes matters: $A_{1,3}$ and $A_{3,1}$ are not the same element. It is up to the reader to remember what each axis of each tensor stands for. This problem is exacerbated in modern machine learning, where tensors have multiple axes with different meanings (batches, channels, etc.), and different operations act on different axes. 

In contrast, we propose \emph{named tensors}, in which each axis has a \emph{name} that describes it and ensures there is no confusion between axes.
We write $\ax[n]$ for an axis with name $\ax$ and size $n$, and we write $\ax(i)$ to index the $i$-th element along axis $\ax$.
So if a tensor has axes $\ax_1[n_1],\ldots,\ax_r[n_r]$ (with $\ax_1,\ldots, \ax_r$ being distinct names), it can be thought of as a map that takes as input a \emph{record} $\{ \ax_1(i_1) ,\ldots, \ax_r(i_r) \}$, with $i_1 \in [n_1], \ldots, i_r \in [n_r]$, and outputs a field element.

In summary the key difference is that, while a tensor in standard notation takes as input an ordered tuple of indices, a named tensor takes as input a record, which is an unordered set of named indices. We illustrate with some examples below, then give formal definitions.

\subsection{By example} \label{sec:example}

For example, if $A$ represents a $3\times 3$ grayscale image, we can make it a named tensor like so (writing it two equivalent ways to show that the order of axes does not matter):

\begin{align}
  A &\in \reals^{\height[3] \times \width[3]} = \reals^{\width[3] \times \height[3]} \notag \\
  A &= \nmatrix{\height}{\width}{
    3 & 1 & 4 \\
    1 & 5 & 9 \\
    2 & 6 & 5
  } = \nmatrix{\width}{\height}{
    3 & 1 & 2 \\
    1 & 5 & 6 \\
    4 & 9 & 5
  }. \label{eq:example_tensor}
\end{align}

We access elements of $A$ using named indices, whose order again does not matter: $A_{\height(1), \width(3)} = A_{\width(3), \height(1)} = 4$.
We also allow partial indexing:
\begin{align*}
A_{\height(1)} &= \nmatrix{}{\width}{
  3 & 1 & 4
}
&
A_{\width(3)} &= \nmatrix{}{\height}{
  4 & 9 & 5
}.
\end{align*}

It does not matter if we write  $A_{\height(1)}$ or $A_{\width(3)}$ as row and column vectors.
In many contexts, an axis name is used with only one size. If so, we can simply write $\height$ for the unique axis with name $\height$, as in $\mathbb{R}^{\height \times \width}$. 
We can leave the size of an axis unspecified at first, and specify its size later (e.g., deferring it to an appendix on experimental details).
For example, we can specify $|\height|=|\width|=28$ if we want to prescribe the precise size of an image, or just write $|\height|=|\width|$ to specify that it's a square image.

\subsection{What's in a name?}
\label{sec:goodnames}

Although users of this notation are free to choose any names for axes, we offer the following recommendations.
First, we recommend \emph{words} instead of single letters, to communicate better the meaning of each axis. 

More subtly, we recommend words that describe a \emph{whole} rather than its parts. For example, to represent a minibatch of examples, we would name the axis $\batch$; to represent a sequence of tokens, we would name the axis $\seq$. 
One reason for this choice is that there are cases, like $\height$ and $\width$, where there is a name for the whole, but no unambiguous name for the part. By contrast, in cases where there is a name for the part but not the whole, it's always possible to use the plural form of the name of the part. For example, if we wanted $A$ to have red, green, and blue channels, we would name the axis $\chans$.

\Cref{sec:examples} contains many more examples of axis names.

\subsection{Formal definition}

We now define formally the notation we use.

\begin{definition}[Names, indices, and axes]
An \emph{axis} is a pair, written $\ax[I]$, where
\begin{itemize}
\item $\ax$ is the \emph{name} of the axis, which is simply a string of letters.
We write both names and variables ranging over names using sans-serif font.
\item $I$ is a set of \emph{indices}. In this paper, $I$ is always of the form $\{1, \ldots, n\}$ for some $n$, so we abbreviate $\ax[\{1, \ldots, n\}]$ as $\ax[n]$.
\end{itemize}
In many contexts, there is only one axis with name $\ax$, and so we refer to the axis simply as $\ax$. The context always makes it clear whether $\ax$ is a name or an axis. If $\ax$ is an axis, we write $\ind(\ax)$ for its index set, and we write $|\ax|$ as shorthand for~$|\ind(\ax)|$.
\end{definition}

\begin{definition}[Named indices and records]
If $\ax[I]$ is an axis and $i\in I$, then a \emph{named index} is a pair, written $\ax(i)$. 
A \emph{record} is a set of named indices $\{\ax_1(i_1), \ldots, \ax_r(i_r)\}$, where $\ax_1, \ldots \ax_r$ are pairwise distinct names. 
\end{definition}

\begin{definition}[Shapes]
A \emph{shape} is a set of axes, written $\ax_1[I_1] \times \cdots \times \ax_r[I_r]$, where $\ax_1, \ldots \ax_r$ are pairwise distinct names. We write $\emptyset$ for the empty shape. A shape defines a set of records:
\begin{equation*}
\rec (\ax_1[I_1] \times \cdots \times \ax_r[I_r]) = \left\{\{\ax_1(i_1), \ldots, \ax_r(i_r)\} \mid i_1 \in I_1, \ldots, i_r \in I_r\right\}.
\end{equation*}
%
We say two shapes $\mathcal{S}$ and $\mathcal{T}$ are \emph{compatible} if whenever $\ax[I] \in \mathcal{S}$ and $\ax[J] \in \mathcal{T}$, then $I = J$. We say that $\mathcal{S}$ and $\mathcal{T}$ are \emph{orthogonal} if there is no $\ax$ such that $\ax[I] \in \mathcal{S}$ and $\ax[J] \in \mathcal{T}$ for any $I$, $J$.
%
If $t \in \rec \mathcal{T}$ and $\mathcal{S} \subseteq \mathcal{T}$, then we write $\restrict{t}{\mathcal{S}}$ for the unique record in $\rec \mathcal{S}$ such that $\restrict{t}{\mathcal{S}} \subseteq t$.
\end{definition}

\begin{definition}[Named tensors]
Let $F$ be a field and let $\mathcal{S}$ be a shape. Then a \emph{named tensor over $F$ with shape $\mathcal{S}$} is a mapping from $\rec \mathcal{S}$ to $F$. If $X$ has shape $\mathcal{S}$ then we write $\shp X = \mathcal{S}$. We write the set of all named tensors with shape $\mathcal{S}$ as $F^{\mathcal{S}}$.
\end{definition}

We don't make any distinction between a scalar (an element of $F$) and a named tensor with empty shape (an element of $F^\emptyset$).

If $X \in F^{\mathcal{S}}$, then we access an element of $X$ by applying it to a record $s \in \rec \mathcal{S}$; but we write this using the usual subscript notation: $X_s$ rather than $X(s)$. To avoid clutter, in place of $X_{\{\ax_1(i_1), \ldots, \ax_r(i_r)\}}$, we usually write $X_{\ax_1(i_1), \ldots, \ax_r(x_r)}$. When a named tensor is an expression like $(X+Y)$, we index it by surrounding it with square brackets like this: $[X+Y]_{\ax_1(i_1), \ldots, \ax_r(x_r)}$.

We also allow partial indexing. If $X$ is a tensor with shape $\mathcal{T}$ and $s \in \rec \mathcal{S}$ where $\mathcal{S} \subseteq \mathcal{T}$, then we define $X_s$ to be the named tensor with shape $\mathcal{T} \setminus \mathcal{S}$ such that, for any $t \in \rec (\mathcal{T} \setminus \mathcal{S})$,
\begin{align*}
\left[X_s\right]_t &= X_{s \cup t}.
\end{align*}
(For the edge case $\mathcal{T} = \emptyset$, our definitions for indexing and partial indexing coincide: one gives a scalar and the other gives a tensor with empty shape, but we don't distinguish between the two.)



\section{Operations}
\label{sec:operations}
A significant benefit of named tensor notation is that it allows one to unambiguously specify \emph{operations} that map tensors to tensors, and defines precisely how operations can be \emph{\liftedVBN} when an 
operation is applied to tensors with more axes than are present in its signature and how \emph{broadcasting} happens when different arguments add different axes.

We start with the formal definition of named tensor operations and lifting, then show how this definition leads to many common operations.

\subsection{Formal definition}

By \emph{(named) tensor function} or \emph{(named) tensor operation}, we mean not only functions from tensors to tensors, but also operators like negation ($-$), addition ($+$), and so on. We will extend the standard function/operator notation by allowing tensor operations to be \emph{\liftedVBN} to higher-order tensors.

\begin{definition}[\liftingNN, unary] \label{def:lifting}
Let $f \colon F^{\mathcal{S}} \rightarrow G^{\mathcal{T}}$ be a function from tensors to tensors. For any shape $\mathcal{S'}$ orthogonal to both $\mathcal{S}$ and $\mathcal{T}$, we can define the \emph{\liftNN} $f^{\mathcal{S'}}$ of $f$ with the shape $\mathcal{S'}$ to be the map
\begin{align*}
f^{\mathcal{S'}} \colon F^{\mathcal{S} \cup \mathcal{S'}} &\rightarrow G^{\mathcal{T} \cup \mathcal{S'}} \\
\left[f^{\mathcal{S'}}(X)\right]_{s'} &= f(X_s) \qquad \text{for all $A \in F^{\mathcal{S}\cup\mathcal{S'}}$ and  $s' \in \rec\mathcal{S'}$.}
\end{align*}
Usually, we simply write $f$ instead of $f^{\mathcal{S'}}$. That is, for every tensor $X$ with shape  $\mathcal{R} \supseteq \mathcal{S}$, we let $f(X) = f^{\mathcal{R} \setminus \mathcal{S}}(X)$.
\end{definition}

If $f$ is a multary function, we can \liftVB{} each of its arguments to larger shapes, and we don't have to add the same axes to all the arguments; an axis present in one argument but not another is \emph{broadcast} from the former to the latter. We consider just the case of two arguments; three or more arguments are analogous. 
\begin{definition}[\liftingNN, binary] \label{def:lifting2}
Let $f \colon F^{\mathcal{S}} \times G^{\mathcal{T}} \rightarrow H^{\mathcal{U}}$ be a binary function from tensors to tensors. For any shapes $\mathcal{S'}$ and $\mathcal{T'}$ that are compatible with each other and orthogonal to $\mathcal{S}$ and $\mathcal{T}$, respectively, and such that $\mathcal{S'} \cup \mathcal{T'}$ is orthogonal to $\mathcal{U}$, we can \liftVB{} $f$ to:
\begin{align*}
f^{\mathcal{S'} \cup \mathcal{T'}} \colon F^{\mathcal{S} \cup \mathcal{S'}} \times G^{\mathcal{T} \cup \mathcal{T'}} &\rightarrow H^{\mathcal{U} \cup \mathcal{S'} \cup \mathcal{T'}} \\
  \left[f^{\mathcal{S'} \cup \mathcal{T'}}(X,Y)\right]_{s'} &= f\left(X_{\restrict{s}{\mathcal{S'}}},Y_{\restrict{s}{\mathcal{T'}}}\right) \qquad \text{for all $X \in F^{\mathcal{S} \cup \mathcal{S'}}$, $Y \in F^{\mathcal{T} \cup \mathcal{T'}}$, $s' \in \rec (\mathcal{S'} \cup \mathcal{T'})$.}
\end{align*}
Again, we usually write $f$ instead of $f^{\mathcal{S'} \cup \mathcal{T'}}$.
\end{definition}

In the following sections, we present some consequences of the above lifting rules. In particular, we show how they allow one to lift some common operations from operating on scalars, vectors, or matrices to operating on tensors with more axes, and how they correspond to vectorizing and broadcasting (as implemented by NumPy and related packages).

\subsection{Elementwise operations and broadcasting}

Any function from a scalar to a scalar corresponds to a tensor function with signature $F^{\emptyset} \rightarrow F^{\emptyset}$. 
Hence lifting it to any tensor shape, by \cref{def:lifting}, corresponds to elementwise application. 
For example, given the logistic sigmoid function,
\begin{align*}
  \sigma \colon \reals &\rightarrow \reals \\
  \sigma(x) &= \frac{1}{1+\exp(-x)}
\end{align*}  
we can \liftVB{} it to tensors. If $A \in \reals^{\height[3] \times \width[3]}$ is the example tensor (\ref{eq:example_tensor}), then
\begin{equation*}
\sigma(A) = \nmatrix{\height}{\width}{
  \frac1{1+\exp(-3)} & \frac1{1+\exp(-1)} & \frac1{1+\exp(-4)} \\[1ex]
  \frac1{1+\exp(-1)} & \frac1{1+\exp(-5)} & \frac1{1+\exp(-9)} \\[1ex]
  \frac1{1+\exp(-2)} & \frac1{1+\exp(-6)} & \frac1{1+\exp(-5)}
}.
\end{equation*}
Similarly for rectified linear units ($\text{relu}(x) = \max(0, x)$), negation, and so on.

Any function with signature $\reals \times \reals \rightarrow \reals$, including binary operators like addition ($\mathord+$), can be applied to two named tensors with the same shape.
But if we apply a binary function or operator to tensors with different shapes, then, by \cref{def:lifting2}, broadcasting applies. For example, let
\begin{align*}
  x &\in \reals^{\height[3]} & y &\in \reals^{\width[3]} \\
  x &= \nmatrix{\height}{}{
    2 \\ 7 \\ 1
  } & 
  y &= \nmatrix{}{\width}{
    1 & 4 & 1
  }.
\end{align*}
(We write $x$ as a column just to make the broadcasting easier to visualize.) Then, to evaluate $A+x$, we effectively replace $x$ with a new tensor with a copy of $x$ for every index of axis $\width$. Likewise for $A+y$:
\begin{align*}
A + x &= \nmatrix{\height}{\width}{
  3+2 & 1+2 & 4+2 \\
  1+7 & 5+7 & 9+7 \\
  2+1 & 6+1 & 5+1
} &
A + y &= \nmatrix{\height}{\width}{
  3+1 & 1+4 & 4+1 \\
  1+1 & 5+4 & 9+1 \\
  2+1 & 6+4 & 5+1
}.
\end{align*}
Similarly for other operations. We write elementwise multiplication (Hadamard product) as $\odot$.

\subsection{Reductions}
\label{sec:reductions}

The same rules apply to functions from vectors to scalars, called \emph{reductions}. We specify which axis a reduction applies to using a subscript (equivalent to the \verb|axis| argument in NumPy and \verb|dim| in PyTorch), so that even after lifting, we know which axis to reduce.
%
For example, we can define summation:
\begin{align*}
\nsum{\ax} \mathord- \colon \reals^{\ax[I]} &\rightarrow \reals \\
\nsum{\ax} X &= \sum_{i \in I} X_{\ax(i)}
\end{align*}
and use it on $A$ from \cref{eq:example_tensor}:
\begin{align*}
\nsum{\height} A &= \sum_i A_{\height(i)} = \nmatrix{}{\width}{
  3+1+2 & 1+5+6 & 4+9+5
}
\\
\nsum{\width} A &= \sum_j A_{\width(j)} = \nmatrix{}{\height}{
  3+1+4 & 1+5+9 & 2+6+5
}.
\end{align*}

We can also write multiple names to sum over multiple axes:
\begin{equation*}
  \nsum{\height \\ \width} A = \sum_i \sum_j A_{\height(i),\width(j)} = 3+1+4+1+5+9+2+6+5.
\end{equation*}
But a summation with an index variable (like $i$ or $j$ above) is a standard summation over values of that variable, and a summation with no subscript is a standard summation over a set.

Other examples of reductions include:
\begin{align*}
  \nfun{\ax}{norm} X &= \sqrt{\nsum{\ax} X^2} & \nfun{\ax}{norm_\mathit{p}} X &= \biggl(\nsum{\ax} X^p\biggr)^{\frac1p} \\
  \nfun{\ax}{min} X &= \min \{X_{\ax(i)} \mid i \in I\} &
  \nfun{\ax}{max} X &= \max \{X_{\ax(i)} \mid i \in I\} \\
  \nfun{\ax}{mean} X &= \frac{1}{|\ax|} \nsum{\ax} X &
  \nfun{\ax}{var} X &= \frac{1}{|\ax|} \nsum{\ax} (X - \nfun{\ax}{mean} X)^2
\end{align*}

\subsection{Contraction}

The vector dot-product is a function from \emph{two} vectors to a scalar. We write it as follows:
\begin{align*}
\mathord- \ndot{\ax} \mathord- \colon \reals^{\ax[I]} \times \reals^{\ax[I]} &\rightarrow \reals \\
  X \ndot{\ax} Y &= \sum_{i\in I} X_{\ax(i)} Y_{\ax(i)}
\end{align*}
When lifted to higher-order tensors, the dot-product generalizes to the ubiquitous \emph{contraction} operator, which can also be thought of as elementwise multiplication followed by summation over one axis, that is,
\begin{equation}
X \ndot\ax Y = \nsum\ax X \odot Y. \label{eq:ndot_as_nsum}
\end{equation}
For example,
\begin{align*}
A \ndot{\height} x &= \nsum\height A \odot x = \nmatrix{}{\width}{
  3\cdot2 + 1\cdot7 + 2\cdot1 & 1\cdot2 + 5\cdot7 + 6\cdot1 & 4\cdot2 + 9\cdot7 + 5\cdot 1
} \\
A \ndot{\width} y &= \nsum\width A \odot y = \nmatrix{\height}{}{
  3\cdot1 + 1\cdot4 + 4\cdot1 \\
  1\cdot1 + 5\cdot4 + 9\cdot1 \\
  2\cdot1 + 6\cdot4 + 5\cdot1
}.
\end{align*}

Again, we can write multiple names to contract multiple axes at once:
\begin{align*}
A \ndot{\height\\\width} A = \nsum{\height\\\width} A \odot A = 3\cdot3 + 1\cdot1 + 4\cdot4 + 1\cdot1 + 5\cdot5 + 9\cdot9 + 2\cdot2 + 6\cdot6 + 5\cdot5
\end{align*}

A $\odot$ with no axis name under it contracts zero axes and is equivalent to elementwise multiplication, which is the reason we use the same symbol $\odot$ for elementwise multiplication and contraction.
%
The contraction operator can be used for many multiplication-like operations:
\begin{align*}
  x \ndot{\height} x &= \nsum\height x \odot x && \text{inner product} \\ \phantom{\sum_i}
  %[x \odot y]_{\height(i), \width(j)} &= x_{\height(i)} \, y_{\width(j)} && \text{outer product} \\
  x \odot y &= \nmatrix{\height}{\width}{2 \cdot 1 & 2 \cdot 4 & 2 \cdot 1 \\ 7 \cdot 1 & 7 \cdot 4 & 7 \cdot 1 \\ 1 \cdot 1 & 1 \cdot 4 & 1 \cdot 1} && \text{outer product} \\
  A \ndot{\width} y &= \nsum\width A \odot y && \text{matrix-vector product} \\
  x \ndot{\height} A &= \nsum\height x \odot A && \text{vector-matrix product} \\
  A \ndot{\width} B &= \nsum\width A \odot B && \text{matrix-matrix product}~(B \in \reals^{\width \times \width'})
\end{align*}

A contraction of three more tensors can be written as a sum. For example, the three-way inner product of vectors $x, y, z \in \reals^{\width}$ can be written as $\nsum{\width} x \odot y \odot z$.

Like the dot-product from which it is lifted, but unlike matrix multiplication, the contraction operator is commutative, but not associative.
However, contraction does obey the following associative-like law.
\begin{align}
  X \ndot{\mathcal{S} \cup \mathcal{T}} (Y \ndot{\mathcal{U}} Z) &= (X \ndot{\mathcal{S}} Y) \ndot{\mathcal{T}\cup\mathcal{U}} Z && \text{if $\mathcal{S} \cap \shp Z = \mathcal{U} \cap \shp X = \emptyset$}. \label{eq:ndot_assoc2} \\
\intertext{The special case}  
  X \ndot{\mathcal{S}} (Y \ndot{\mathcal{U}} Z) &= (X \ndot{\mathcal{S}} Y) \ndot{\mathcal{U}} Z && \text{if $\mathcal{S} \cap \shp Z = \mathcal{U} \cap \shp X = \emptyset$} \label{eq:ndot_assoc}
\end{align}
will be useful in \cref{sec:calculus} for moving $Z$ from inside one or more sets of parentheses to the outside.

\subsection{Vectors to vectors and beyond}

Functions from vectors to vectors ($\reals^{\ax[I]} \rightarrow \reals^{\ax[I]}$) lift to functions on tensors that operate along one axis, but leave the tensor shape unchanged. Such functions are particularly problematic in standard notation, which does not provide any way (to our knowledge) of specifying which axis the operation should be performed over. Such functions include:
\begin{subequations}
\begin{align}
  \nfun{\ax}{softmax} X &= \frac{\exp X}{\nsum{\ax} \exp X} \label{eq:def_softmax} \\
  \nfun{\ax}{argmax} X &= \lim_{\alpha \rightarrow \infty} \nfun{\ax}{softmax} \alpha X \\
  \nfun{\ax}{argmin} X &= \lim_{\alpha \rightarrow -\infty} \nfun{\ax}{softmax} \alpha X
\end{align}
\end{subequations}
For example, we can clearly distinguish between two ways of performing a softmax on $A$:
\begin{align*}
  \nfun{\height}{softmax} A &= \nmatrix{\height}{\width}{ 
    \frac{\exp 3}{\exp 3 + \exp 1 + \exp 2} & \frac{\exp 1}{\exp 1 + \exp 5 + \exp 6} & \frac{\exp 4}{\exp 4 + \exp 9 + \exp 5} \\
    \frac{\exp 1}{\exp 3 + \exp 1 + \exp 2} & \frac{\exp 5}{\exp 1 + \exp 5 + \exp 6} & \frac{\exp 9}{\exp 4 + \exp 9 + \exp 5} \\
    \frac{\exp 2}{\exp 3 + \exp 1 + \exp 2} & \frac{\exp 6}{\exp 1 + \exp 5 + \exp 6} & \frac{\exp 5}{\exp 4 + \exp 9 + \exp 5} \\
    }
  \\
  \nfun{\width}{softmax} A &= \nmatrix{\height}{\width}{
    \frac{\exp 3}{\exp 3 + \exp 1 + \exp 4} & \frac{\exp 1}{\exp 3 + \exp 1 + \exp 4} & \frac{\exp 4}{\exp 3 + \exp 1 + \exp 4} \\
    \frac{\exp 1}{\exp 1 + \exp 5 + \exp 9} & \frac{\exp 5}{\exp 1 + \exp 5 + \exp 9} & \frac{\exp 9}{\exp 1 + \exp 5 + \exp 9} \\
    \frac{\exp 2}{\exp 2 + \exp 6 + \exp 5} & \frac{\exp 6}{\exp 2 + \exp 6 + \exp 5} & \frac{\exp 5}{\exp 2 + \exp 6 + \exp 5} \\
    }
\end{align*}

\subsection{Renaming and reshaping}

It's often useful to rename an axis (analogous to a transpose operation in standard notation). We can think of this as the lift of a function from vectors to vectors, but with different input and output axes:
\begin{align*}
[\mathord-]_{\ax\rightarrow\ax'} \colon \reals^{\ax[I]} &\rightarrow \reals^{\ax'[I]} \\
[X_{\ax\rightarrow\ax'}]_{\ax'(i)} &= X_{\ax(i)}
\end{align*}
For example,
\begin{equation*}
A_{\height\rightarrow\height'} = \nmatrix{\height'}{\width}{
  3 & 1 & 4 \\
  1 & 5 & 9 \\
  2 & 6 & 5 \\
}.
\end{equation*}
We can also define notation for reshaping two or more axes into one axis:
\begin{equation*}
  A_{(\height,\width)\rightarrow\layer} = \nmatrix{}{\layer}{
    3 & 1 & 4 & 1 & 5 & 9 & 2 & 6 & 5
  }
\end{equation*}
%Similarly, we can reshape one axis into two or more axes, or even multiple axes into multiple axes.
The order of elements in the new axis is undefined. Authors who need a particular ordering may write a more specific definition.

\subsection{Indexing\protect\footnotemark}

\footnotetext{We are grateful to Tongfei Chen and Chu-Cheng Lin for contributing the original idea behind this section, as well as the example.}

NumPy and its derivatives provide various ways to recombine elements of a tensor to form a new tensor: integer array indexing, and functions like \verb|numpy.take|, \verb|numpy.take_along_axis|, \verb|torch.index_select|, and \verb|torch.gather|. Using named tensors, we can write nearly all of these operations as lifts of a single function:
\begin{align*}
  \nfun{\ax}{index} \colon \reals^{\ax[n]} \times [n] &\rightarrow \reals \\
  \nfun{\ax}{index}(X, i) &= X_{\ax(i)}.
\end{align*}
For example, suppose we have
\begin{align*}
  E &\in \reals^{\vocab[n] \times \emb} \\
  i &\in [n] \\
  I &\in [n]^{\seq} \\
  P &\in \reals^{\seq \times \vocab[n]}
\end{align*}
Tensor~$E$ contains word embeddings for all the words in the vocabulary. Integer~$i$ is the numeric identifier of a word, while tensor~$I$ is a sequence of numeric identifiers of words. Tensor~$P$ contains a sequence of probability distributions over the vocabulary (e.g., the predictions of a language model). Then:
\begin{itemize}
\item $\nfun{\vocab}{index}(E,i)$ broadcasts the $\emb$ axis from $E$ to $i$, giving the word embedding of word $i$. This is the same as partial indexing ($E_{\vocab(i)}$).
\item $\nfun{\vocab}{index}(E,I)$ also broadcasts the $\seq$ axis from $I$ to $E$, giving a sequence of word embeddings. This is the same as integer array indexing (\texttt{$E$[$I$]}), \texttt{numpy.take($E$, $I$, 0)}, or \texttt{torch.index\_select($E$, 0, $I$)}.
\item $\nfun{\vocab}{index}(P,I)$ aligns $P$'s and $I$'s $\seq$ axes, giving a sequence of probabilities. This is the same as \texttt{numpy.take\_along\_axis($P$, $I$, 0)} or \texttt{torch.gather($P$, 0, $I$)}.
\item If $P$ and $I$ additionally had a $\batch$ axis (before the other axes), then $\nfun{\vocab}{index}(P,I)$ would be the same as \texttt{tensorflow.gather($P$, $I$, axis=1, batch\_dims=1)}.
\end{itemize}

In NumPy, indexing using two or more integer arrays requires a special definition with some surprising special cases. With named tensors, we simply apply the indexing function twice. For example, if we wanted to get probabilities of words $J$ at a subset $I$ of positions:
\begin{align*}
  |\seq| &= m \\
  I &\in [m]^\subseq && \text{positions} \\
  J &\in [n]^\subseq && \text{numeric identifiers} \\
  S &= \nfun{\vocab}{index}(\nfun{\seq}{index}(P, I), J) \in \reals^{\subseq} \\
\intertext{so that}  
  S_{\subseq(k)} &= P_{\seq(I_{\subseq(k)}), \vocab(I_{\subseq(k)})}.
\end{align*}




\section{Worked Examples: Neural Networks}
\label{sec:examples}
In this section we give a series of worked examples illustrating how standard neural network model components can be written using named tensors. Appendix~\ref{sec:examples_long} builds some of these components into complete specifications of the Transformer and LeNet.

\subsection{Feedforward neural networks}

A multi-layer, feedforward neural network with different-sized layers can be written as:
\begin{align*}
  X^0 &\in \mathbb{R}^{\inp} \\
  X^1 &= \sigma(W^1 \ndot{\inp} X^0 + b^1) & W^1 &\in \mathbb{R}^{\hidden_1 \times \inp} & b^1 &\in \mathbb{R}^{\hidden_1} \\
  X^2 &= \sigma(W^2 \ndot{\hidden_1} X^1 + b^2) & W^2 &\in \mathbb{R}^{\hidden_2 \times \hidden_1} & b^2 &\in \mathbb{R}^{\hidden_2} \\
  X^3 &= \sigma(W^3 \ndot{\hidden_2} X^2 + b^3) & W^3 &\in \mathbb{R}^{\out \times \hidden_2} & b^3 &\in \mathbb{R}^{\out}
\end{align*}
The layer sizes can be specified by writing $|\hidden_1| = n_1$, etc. As noted above, $\sigma$ is applied elementwise and does not require additional annotation. 

Alternatively, the layer equation can be abstracted by defining:
\begin{align*}
  \text{FullConn}^l(x) &= \sigma\Bigl(W^l \ndot{\layer} x + b^l\Bigr)_{\layer'\rightarrow\layer}
\end{align*}
where
\begin{align*}
  W^l &\in \mathbb{R}^{\layer'[n_l] \times \layer[n_{l-1}]} \\
  b^l &\in \mathbb{R}^{\layer'[n_l]}.
\end{align*}
The function $\text{FullConn}^l$ encapsulates both the equation for layer $l$ as well as its parameters $W^l, b^l$ (analogous to what TensorFlow and PyTorch call \emph{modules}). Since we chose to use the same axis name $\layer$ for all the layers (with different sizes $n_l$), $\text{FullConn}^l$ temporarily computes its output over axis $\layer'$, then renames it back to $\layer$. The network can be defined like this:
\begin{align*}
  X^0 &\in \mathbb{R}^{\layer[n_0]} \\
  X^1 &= \text{FullConn}^1(X^0) \\
  X^2 &= \text{FullConn}^2(X^1) \\
  X^3 &= \text{FullConn}^3(X^2).
\end{align*}

\subsection{Recurrent neural networks}
\label{sec:rnn}

As a second example, we consider a simple (Elman) RNN. This model is similar to the feedforward network, except that the number of timesteps is variable and parameters are shared over time. At each time step, it produces a tensor with a new axis $\hidden'$ which is then renamed $\hidden$ for the next step in the recursion. 
\begin{align*}
x^{t} &\in \mathbb{R}^{\inp} & t &= 1, \ldots, n \\
W^{\text{h}} &\in \mathbb{R}^{\hidden \times \hidden'} & |\hidden| &= |\hidden'| \\
W^{\text{i}} &\in \mathbb{R}^{\inp \times \hidden'} \\
b &\in \mathbb{R}^{\hidden'} \\
h^{0} &\in \mathbb{R}^{\hidden} \\
h^{t} &= \sigma\Bigl( W^{\text{h}} \ndot{\hidden} h^{t-1} + W^{\text{i}} \ndot{\inp} x^{t} + b \Bigr)_{\hidden'\rightarrow\hidden} & t &= 1, \ldots, n
\end{align*}

\subsection{Attention}
\label{sec:attention}

In the introduction (\S\ref{sec:intro}), we described difficulties in interpreting the equation for attention as used with Transformers~\citep{vaswani+:2017}. In our notation, it looks like this:
\begin{align}
  \text{Attention} \colon \mathbb{R}^{\key} \times \mathbb{R}^{\seq \times\key} \times \mathbb{R}^{\seq \times\val} &\rightarrow \mathbb{R}^{\val} \\
  \text{Attention}(Q,K,V) &= \left( \nfun{\seq}{softmax} \frac{Q \ndot{\key} K}{\sqrt{|\key|}} \right) \ndot{\seq} V. \label{eq:def_att}
\end{align}

This definition takes a single query $Q$ vector and returns a single result vector (and actually could be further reduced to a scalar values as $\val$ is not strictly necessary). To apply to a sequence, we can give $Q$ a $\seq'$ axis, and the function will compute an output sequence. Providing $Q$, $K$, and $V$ with a $\heads$ axis lifts the function to compute multiple attention heads. 

For Transformers we often need to apply a mask to ensure attention is only applied to 
valid keys (e.g. for causal language models). We can modify the equation to include this mask:
\begin{align*}
  \text{Attention} \colon \mathbb{R}^{\key} \times \mathbb{R}^{\seq \times\key} \times \mathbb{R}^{\seq \times\val} \times \mathbb{R}^{\seq} &\rightarrow \mathbb{R}^{\val} \\
\text{Attention}(Q, K, V, M) &= \left( \nfun{\seq}{softmax} \left( \frac{Q \ndot{\key} K}{\sqrt{|\key|}} + M \right) \right) \ndot{\seq} V.
\end{align*}

Appendix~\ref{sec:transformer} includes a full specification of the complete Transformer model using the named tensor notation. 

\subsection{Convolution}

Standard neural network convolutions can be specified by ``unrolling'' a vector and then applying a standard dot product. We define an axis-parameterized unrolling function that converts a one-axis tensor to a sequence of $\kernel$ sized vectors:
\begin{align*}
  \nfun{\seq \\ \kernel}{unroll} \colon \reals^{\seq[n]} &\rightarrow \reals^{\seq[n-|\kernel|+1], \kernel} \\
  \nfun{\seq \\ \kernel}{unroll} X &= Y,\ \text{where} \\
  Y_{\seq(i), \kernel(j)} &= X_{\seq(i+j - 1)}.
\end{align*}

A 1d convolution with input channels $\chans$ and output channels $\chans'$ consists of unrolling along the $\seq$ axis and then taking a dot product: 
\begin{align*}
\text{Conv1d} \colon \reals^{\chans \times \seq[n]} &\rightarrow \mathbb{R}^{\chans' \times \seq[n']} \\
\text{Conv1d}(X; W, b) &= W \ndot{\chans \\ \kernel} \nfun{\seq \\ \kernel}{unroll} X + b
\end{align*}
where
\begin{align*}
W &\in \reals^{\chans' \times \chans \times \kernel} \\
b &\in \reals^{\chans'} \\
\end{align*}

Unrolling easily generalizes to higher-dimensional convolutions:
\begin{align*}
  \text{Conv2d} \colon \reals^{\chans \times \height[h] \times \width[w]}
  &\rightarrow \reals^{\chans' \times \height[h'] \times \width[w']} \\
  \text{Conv2d}(X; W, b) &= W \ndot{\chans \\ \kh, \kw} \nfun{\height \\ \kh}{unroll} \nfun{\width\\\kw}{unroll} X + b
\end{align*}  
where
\begin{align*}
W &\in \reals^{\chans' \times \chans \times \kh \times \kw} \\
b &\in \reals^{\chans'}.
\end{align*}

\subsection{Pooling}

Pooling is similar to convolutions. 
We first define a function to partition a tensor into windows. 
\begin{align*}
  \nfun{\seq,\kernel}{pool} \colon \reals^{\seq[n]} &\rightarrow \reals^{\seq[n/|\kernel|],\kernel} \\
  \nfun{\seq,\kernel}{pool} X &= Y,\ \text{where} \\
  Y_{\seq(i), \kernel(j)} &= X_{\seq((i-1) \cdot |\kernel| + j)}.
\end{align*}

Then we can define aggregations over $\kernel$. We define max-pooling as: 
\begin{align*}
\text{MaxPool1d}_{k} \colon \mathbb{R}^{\seq[n]} &\rightarrow \mathbb{R}^{\seq[n/k]} \\
\text{MaxPool1d}_{k}(X) &= \nfun{\kernel}{max} \nfun{\seq,\kernel}{pool} X \\
|\kernel| &= k \\
\text{MaxPool2d}_{kh,kw} \colon \mathbb{R}^{\height[h] \times \width[w]} &\rightarrow \mathbb{R}^{\height[h/kh] \times \width[w/kw]} \\
\text{MaxPool2d}_{kh,kw}(X) &= \nfun{\kh,\kw}{max} \nfun{\height,\kh}{pool} \nfun{\width,\kw}{pool} X \\
|\kh| &= kh \\
|\kw| &= kw.
\end{align*}

\subsection{Normalization layers}

Normalization layers are used in all large-scale deep learning models, with different architectures requiring different types of normalization. However, despite their importance, the differences between them are often not clearly communicated. For example, the PyTorch documentation \citep{pytorchdoc} describes all of them using the same equation (where $\epsilon > 0$ is a small constant for numerical stability): 
\begin{align*}
   Y = \frac{X - \operatorname{mean}(X)}{\sqrt{\operatorname{var}(X) + \epsilon}} \odot \gamma + \beta
\end{align*}
\citet{wu+he:2018} give essentially the same equation and explain the differences using a combination of equations, words, and pictures. But they do not capture differences in $\gamma$ and $\beta$ among different normalization layers.

Critically, the layers do differ by which axes are \textit{standardized} as well as their parameters. We define a single named standardization function as:
\begin{align*}
  \nfun{\ax}{standardize} \colon \mathbb{R}^{\ax} &\rightarrow \mathbb{R}^{\ax} \\
  \nfun{\ax}{standardize}(X) &= \frac{X - \nfun{\ax}{mean}(X)}{\sqrt{\nfun{\ax}{var}(X) + \epsilon}}
\end{align*}


Then, we can define the three kinds of normalization layers, all with type $\reals^{{\batch \times \chans \times \layer}} \rightarrow \reals^{{\batch \times \chans \times \layer}}$. While superficially similar, these functions differ in their standardized axes and their parameter shape. 
\begin{align*}
\text{BatchNorm}(X; \gamma, \beta) &= \nfun{\batch,\layer}{standardize}(X) \ndot{} \gamma + \beta & \gamma, \beta &\in \mathbb{R}^{\chans} \\
\text{InstanceNorm}(X; \gamma, \beta) &= \nfun{\layer}{standardize}(X) \ndot{} \gamma + \beta & \gamma, \beta &\in \mathbb{R}^{\chans} \\
\text{LayerNorm}(X; \gamma, \beta) &= \nfun{\layer,\chans}{standardize}(X) \ndot{} \gamma + \beta & \gamma, \beta &\in \mathbb{R}^{\chans,\layer}
\end{align*}

\iffalse
Other deep learning methods have been proposed which consider different shapes of standardization. For instance, group norm is a popular extension that first pools channels into $k$-size groups before standardizing. 

\begin{align*}
\text{GroupNorm}_k(X; \gamma, \beta) &= \left[ \nfun{\kernel,\layer}{standardize} \nfun{\chans, \kernel}{pool} X \right]_{(\chans,\kernel)\rightarrow \chans } \ndot{} \gamma + \beta \\ 
\end{align*}
where
\begin{align*}
|\kernel| &= k\\
\gamma, \beta &\in \mathbb{R}^{\chans}.
\end{align*}
\fi

\iffalse
\subsection{Other examples}

\subsubsection{Discrete random variables}

Named axes are very helpful for working with discrete random variables, because each random variable can be represented by an axis with the same name. For instance, if $\name{A}$ and $\name{B}$ are random variables, we can treat $p(\name{B} \mid \name{A})$ and $p(\name{A})$ as tensors:
\begin{align*}
p(\name{B} \mid \name{A}) &\in [0, 1]^{\name{A} \times \name{B}} & \nsum{\name{B}} p(\name{B} \mid \name{A}) &= 1 \\
p(\name{A}) &\in [0, 1]^{\name{A}} & \nsum{\name{A}} p(\name{A}) &= 1
\end{align*}
Then many common operations on probability distributions can be expressed in terms of tensor operations:
\begin{align*}
p(\name{A}, \name{B}) &= p(\name{B} \mid \name{A}) \odot p(\name{A}) && \text{chain rule}\\
p(\name{B}) &= \nsum{\name{A}} p(\name{A}, \name{B}) = p(\name{B} \mid \name{A}) \ndot{\name{A}} p(\name{A}) && \text{marginalization} \\
p(\name{A} \mid \name{B}) &= \frac{p(\name{A}, \name{B})}{p(\name{B})} = \frac{p(\name{B} \mid \name{A}) \odot p(\name{A})}{p(\name{B} \mid \name{A}) \ndot{\name{A}} p(\name{A})}. && \text{Bayes' rule}
\end{align*}

\subsubsection{Continuous bag of words}

A continuous bag-of-words model classifies by summing up the embeddings of a sequence of words $X$ (as one-hot vectors) and projecting them to the space of classes. 

\begin{align*}
\text{CBOW} \colon \{0, 1\}^{\seq \times \vocab} &\rightarrow \reals^{\classes} \\
\text{CBOW}(X; E, W) &= \nfun{\classes}{softmax} \left(\nsum{\seq} W \ndot{\emb} E \ndot{\vocab} X\right)
\end{align*}
where
\begin{align*}
\nsum{\vocab} X_{\seq(i)} &= 1 & i &= 1, \ldots, |\seq| \\
E &\in \reals^{\vocab \times \emb} \\
W &\in \reals^{\classes \times \emb}.
\end{align*}

\subsubsection{Sudoku ILP}

\ndef{\assign}{assign}

Sudoku puzzles can be represented as  binary tiled tensors.
Given a grid we can check that it is valid by converting it to a grid of grids. 
Constraints then ensure that there is one digit per row, per column and per sub-box.

\begin{align*}
\text{check} \colon \{0, 1\}^{\height[9] \times \width[9] \times \assign[9]} &\rightarrow \{0, 1\} \\
\text{check}(X) &=
\mathbb{I}\left[\begin{aligned}
\nsum{\assign} X = 1 &\land \nsum{\height \\ \width} Y = 1 \land {} \\
\nsum{\height} X = 1 &\land \nsum{\width} X = 1
\end{aligned}\right]
\end{align*}
where
\begin{align*}
Y &\in \{0, 1\}^{\height'[3] \times \width'[3] \times \height[3] \times \width[3] \times \assign[9]}  \\
Y_{\height'(h'), \height(h), \width'(w'), \width(w)} &= X_{\height(3h' + h-1), \width(3 w' + w-1)}.
\end{align*} 

\subsubsection{$K$-means clustering}

\ndef{\clusters}{clusters}

The following equations define one step of $k$-means clustering. Given a set of points $X$ and an initial set of cluster centers $C$,
\begin{align*}
  X &\in \reals^{\batch \times \dd} \\
C &\in \reals^{\clusters \times \dd}
\end{align*}
we repeat the following update: Compute cluster assignments
\begin{align*}
Q &= \nfun{\clusters}{argmin} \nfun{\dd}{norm}(C-X)
\end{align*}
then recompute the cluster centers:
\begin{equation*}
C \leftarrow \nsum{\batch} \frac{Q \odot X}{Q}.
\end{equation*}

\subsubsection{Beam search}

\ndef{\beam}{beam}

Beam search is a commonly used approach for approximate discrete search. Here $H$ is the score of each element in the beam, $S$ is the state of each element in the beam, and $f$ is an update function that returns the score of each state transition. 
\begin{align*} 
H &\in \reals^{\beam} \\
S &\in \{0, 1\}^{\beam \times \state} & \nsum{\state} S &= 1 \\
f &\colon \{0, 1\}^{\state} \rightarrow \reals^{\state} \\
\end{align*}
Then we repeat the following update:
\begin{align*}
H' &= \nfun{\beam}{max} (H \odot f(S)) \\
H &\leftarrow \nfun{\state,\beam}{maxk} H' \\
S &\leftarrow \nfun{\state,\beam}{argmaxk} H'
\end{align*}
where
\begin{align*}
\nfun{\ax,\name{k}}{maxk} \colon \reals^{\ax} &\rightarrow \reals^{\name{k}} \\
\nfun{\ax,\name{k}}{argmaxk} \colon \reals^{\ax} &\rightarrow \{0,1\}^{\ax,\name{k}}
\end{align*}
are defined such that $[\nfun{\ax,\name{k}}{maxk} A]_{\name{k}(i)}$ is the $i$-th largest value along axis $\ax$ and $A \ndot{\ax} (\nfun{\ax,\name{k}}{argmaxk}{A}) = \nfun{\ax,\name{k}}{max} A$.

We can add a $\batch$ axis to $H$ and $S$ and the above equations will work unchanged.

\subsubsection{Multivariate normal distribution}

To define a multivariate normal distribution, we need some matrix operations. These have two axis names written under them, for rows and columns, respectively. Determinant and inverse have the following signatures:
\begin{align*}
\nfun{\ax_1,\ax_2}{det} \colon F^{\ax_1[n] \times \ax_2[n]} &\rightarrow F \\
\nfun{\ax_1,\ax_2}{inv} \colon F^{\ax_1[n] \times \ax_2[n]} &\rightarrow F^{\ax_1[n] \times \ax_2[n]}.
\end{align*}
(We write $\text{inv}$ instead of $\cdot^{-1}$ because there's no way to write axis names under the latter.)

In our notation, the application of a bilinear form is more verbose than the standard notation ($(X-\mu)^\top \Sigma^{-1} (X-\mu)$), but also makes it look more like a function of two arguments (and would generalize to three or more arguments).

\begin{align*}
\mathcal{N} \colon \reals^{\dd} &\rightarrow \reals \\
\mathcal{N}(X; \mu, \Sigma) &= \frac{\exp\left(-\frac{1}{2} \left(\nfun{\dd_1, \dd_2}{inv} \Sigma\right) \ndot{\dd_1,\dd_2} \left([X - \mu]_{\dd\rightarrow\dd_1} \odot [X - \mu]_{\dd\rightarrow\dd_2} \right) \right)}{\sqrt{(2 \pi)^{|\dd|} \nfun{\dd_1, \dd_2}{det} \Sigma}}
\end{align*}
where
\begin{align*}
|\dd| &= |\dd_1| = |\dd_2| \\
\mu &\in \reals^{\dd} \\
\Sigma & \in \reals^{\dd_1 \times \dd_2}.
\end{align*}
\fi


\section{Differential Calculus}
\label{sec:calculus}
\newcommand{\ddx}[1]{\frac{\partial #1}{\partial X}}
\newcommand{\inp}[1]{#1^*}

If $f$ is a function from order-$m$ tensors to order-$n$ tensors, the partial derivatives of $f$ (evaluated on a tensor $X$) form an order-$(m+n)$ tensor: $m$ ``input'' axes for the directions in which $X$ could change and $n$ ``output'' axes for the change in $f(X)$.

For example, the derivative of a function from vectors to vectors is a matrix (the Jacobian). But using matrix notation, there are conflicting conventions about whether the first axis is the input axis (``denominator layout'') or the output axis (``numerator layout''). The derivative of a function from vectors to matrices or matrices to vectors cannot be represented as a matrix at all, so one must resort to flattening the matrices into vectors.

With tensors, taking derivatives of higher-order tensors with respect to higher-order tensors is not difficult \citep{laue+:2018}. With named tensors, we get the additional advantage of using names to distinguish input and output axes.

\subsection{Definition}

Let $f \colon \reals^\mathcal{S} \rightarrow \reals^\mathcal{T}$. The derivative of $f$ (evaluated at $X$) has an input axis for each axis in $\mathcal{S}$ and an output axis for each axis in $\mathcal{T}$, and they have to have distinct names. So if $\mathcal{S} = \name{ax_1} \times \cdots \times \name{ax\sub{r}}$, then for each axis name $\name{ax\sub{i}}$, let $\name{\inp{ax\sub{i}}}$ be a new axis name, not in $\mathcal{T}$, and let $\inp{\mathcal{S}} = \name{\inp{ax_1}} \times \cdots \times \name{\inp{ax\sub{r}}}$. If $s = \{\nidx{ax_1}{i_1}, \ldots, \nidx{ax\sub{r}}{i_r}\}$, let $\inp{s} = \{\nidx{\inp{ax_1}}{i_1}, \ldots, \nidx{\inp{ax\sub{r}}}{i_r}\}$.

Then the derivative of $f$ at $X$ is the tensor with shape $\inp{\mathcal{S}} \times \mathcal{T}$ such that for all $s \in \rec\mathcal{S}$ and $t \in \rec\mathcal{T}$,
\[\left[\ddx f(X) \right]_{\inp{s},t} = \frac{\partial}{\partial X_s} [f(X)]_t.\]

We'll often make use of the following generalization of the identity matrix:
\begin{align*}
  I_\mathcal{S} &\in \reals^{\inp{\mathcal{S}} \times \mathcal{S}} \\
  [I_\mathcal{S}]_{\inp{s}, s} &= \begin{cases}
    1 & \text{if $\inp{s} = s$} \\
    0 & \text{otherwise.}
  \end{cases}
\end{align*}

\subsection{Rules}

Now we give some rules for computing derivatives. Unless otherwise indicated, $X$ has shape $\mathcal{S}$, and $U$ and $V$ are dependent on $x$ and have shapes $\mathcal{U}$ and $\mathcal{V}$, respectively.
\begin{align*}
  \ddx X &= I_\mathcal{S} \\
  \ddx U &= 0 && \text{$U$ does not depend on $X$} \\
  \ddx{f(U)} &= f'(U) \odot \ddx U && f \colon \reals \rightarrow \reals \\
  \ddx{} (U + V) &= \ddx U + \ddx V \\
  \ddx{} \nsum{ax} U &= \nsum{ax} \ddx U \\
  \ddx{} (U \odot V) &= \ddx U \odot V + U \odot \ddx V \\
  \ddx{} (U \ndot{ax} V) &= \ddx U \ndot{ax} V + U \ndot{ax} \ddx V \\
  \ddx{} \frac{U}{V} &= \frac{\ddx U \odot V - U \odot \ddx V}{V^2} \\
  \ddx{U_r} &= \left[\ddx U\right]_r && r \in \rec \mathcal{R}, \mathcal{R} \subseteq \mathcal{U} \\
  \ddx{U_{\nmov{ax1}{ax2}}} &= \left[\ddx U\right]_{\nmov{ax1}{ax2}}
\end{align*}

The chain rule above is for elementwise operations. The general chain rule looks like this for functions of one and two variables; three or more variables are analogous.
\begin{align*}
  \ddx{f(U)} &= \frac{\partial f(U)}{\partial U} \ndot{\inp{\mathcal{U}} \mid \mathcal{U}} \ddx U \\
  \ddx{f(U, V)} &= \frac{\partial f(U, V)}{\partial U} \ndot{\inp{\mathcal{U}} \mid \mathcal{U}} \ddx U + \frac{\partial f(U, V)}{\partial V} \ndot{\inp{\mathcal{V}} \mid \mathcal{V}} \ddx V
\end{align*}
where $\ndot{ax1|ax2}$ contracts $\name{ax1}$ in the left operand with $\name{ax2}$ in the right operand: $A \ndot{ax1|ax2} B = \sum_i A_{\nidx{ax1}{i}} \odot B_{\nidx{ax2}{i}}$.

\subsection{Examples}

Here's an example using these rules to derive the Jacobian for softmax:
\begin{align*}
  Y &= \nfun{ax}{softmax} X \\
  \ddx Y &= \ddx{} \frac{\exp X}{\nsum{ax} \exp X} \\
    &= \frac{\exp X \odot \ddx X \odot \nsum{ax} \exp X - \exp X \odot \nsum{ax} (\exp X \odot \ddx X)}{(\nsum{ax} \exp X)^2} \\
    &= Y \odot \left(\ddx X - Y \ndot{ax} \ddx X\right) \\
    &= Y \odot (I_\name{ax} - Y \ndot{ax} I_\name{ax}) \\
    &= Y \odot (I_\name{ax} - Y_\nmov{ax}{\inp{ax}}).
\end{align*}
To derive the backpropagation rule:
\begin{align*}
  \ddx{f(Y)} &= f'(Y) \ndot{\inp{ax}|ax} \ddx Y \\
  &= f'(Y) \ndot{\inp{ax}|ax} (Y \odot (I_\name{ax} - Y_\nmov{ax}{\inp{ax}})) \\
  &= f'(Y) \ndot{\inp{ax}|ax} (Y \odot I_\name{ax} - f'(Y) \ndot{\inp{ax}|ax} (Y \odot Y_\nmov{ax}{\inp{ax}}) \\
  &= f'(Y) \odot Y_\nmov{ax}{\inp{ax}} - (f'(Y) \ndot{\inp{ax}} Y_\nmov{ax}{\inp{ax}}) \odot Y_\nmov{ax}{\inp{ax}} \\
  &= (f'(Y) - f'(Y) \ndot{\inp{ax}} Y_\nmov{ax}{\inp{ax}}) \odot Y_\nmov{ax}{\inp{ax}}.
\end{align*}

As another example, here are the Jacobian and backpropagation rule for Conv1d:
\begin{align*}
  Y &= \text{Conv1d}(X; W) \\
  \frac{\partial Y}{\partial X} &= W \ndot{chans,kern} \frac{\partial U}{\partial X} \\
  \frac{\partial U_{\nidx{seq}{i},\nidx{kern}{j}}}{\partial X_{\nidx{seq^*}{k}}} &= \delta(i+j-1,k) I_\name{chans} \\
  \frac{\partial f(Y)}{\partial X} &= f'(Y) \ndot{seq^*|seq} \left(W \ndot{chans,kern} \frac{\partial U}{\partial X}\right) \\
  &= \left(W \ndot{chans,kern} \frac{\partial U}{\partial X}\right) \ndot{seq|seq^*} f'(Y) \\
  &= W \ndot{chans,kern} \left( \frac{\partial U}{\partial X} \ndot{seq|seq^*} f'(Y) \right) \\
  &= W \ndot{chans,kern} V \\
  V &= \frac{\partial U}{\partial X} \ndot{seq|seq^*} f'(Y) \\
  V_{\nidx{seq^*}{k},\nidx{kern}{j}} &= \sum_j \delta(i+j-1,k) I_\name{chans} \odot f'(Y)_{\nidx{seq^*}{j}} \\
  &= I_\name{chans} \odot f'(Y)_{\nidx{seq^*}{k-i+1}}.
\end{align*}

\subsection{Broadcasting}

If $f \colon \reals^\mathcal{S} \rightarrow \reals^\mathcal{T}$, then recall that $f$ can be extended to $\reals^{\mathcal{S} \cup \mathcal{S^+}}$ where $\mathcal{S}$ and $\mathcal{S^+}$ are orthogonal.

It's more convenient here to notate the derivative of $f$ as $Df$. If $f$ has two arguments, its partial derivatives are $D_1 f$ and $D_2 f$.

Although $Df$ extends to $\reals^{\mathcal{S} \cup \mathcal{S^+}}$ using the usual broadcasting rules, the extension of the derivative is unfortunately not the derivative of the extension. To avoid confusion, write $f^+$ for the extension:
\begin{align*}
  f^+ \colon \reals^{\mathcal{S} \cup \mathcal{S^+}} &\rightarrow \reals^{\mathcal{T} \cup \mathcal{S^+}} \\
  f^+(X)_r &= f(X_r).
\end{align*}
Then the derivative of $f^+$ is:
\begin{align*}
  Df^+ \colon \reals^{\inp{\mathcal{S}} \cup \inp{\mathcal{{S^+}}} \cup \mathcal{T} \cup \mathcal{S^+}} &\rightarrow \reals^{\mathcal{T} \cup \mathcal{S^+}} \\
  Df^+(X) &= Df(X) \odot I_{\mathcal{S}^+}.
\end{align*}  

Similarly, if $f \colon \reals^\mathcal{S} \times \reals^\mathcal{T} \rightarrow \reals^\mathcal{U}$, we can extend $f$ to $f^+ \colon \reals^\mathcal{S \cup S^+} \times \reals^\mathcal{T \cup T^+} \rightarrow \reals^\mathcal{U \cup S^+ \cup T^+}$. Then
\begin{align*}
  D_1 f^+(X, Y) &= D_1 f(X, Y) \odot I_{\mathcal{S}^+} \\
  D_2 f^+(X, Y) &= D_2 f(X, Y) \odot I_{\mathcal{T}^+}.
\end{align*}



\section{Alternatives and Related Work}
\subsection{Code}

As mentioned in the introduction, our notation is influenced by NumPy and various proposals to implement named axes with NumPy or equivalent libraries in various programming languages. But our proposal here is for mathematical notation to be used in papers and software documentation.

Some might ask, why not simply write code into our papers? For the same reasons that we don't want code to serve as its own documentation. Math notation is more human-readable (in our opinion) than code, abstracting away from details and optimizations that would hinder understanding of the main ideas. It also provides a specification that the implementation can be checked against.

\subsection{Index notation}

A very frequently asked question is why we haven't used index notation as used in physics, and the Einstein summation convention in particular. In this notation, axes are ordered, and every equation is written in terms of tensor components.
If an index appears on both sides of an equation, then the equation must hold for each value of the index, and if an index appears twice on one side and not on the other, there is an implicit summation over that index.
\begin{align*}
  \text{Attention} \colon \reals^{n' \times d_k} \times \reals^{n \times d_k} \times \reals^{n \times d_v} &\rightarrow \reals^{n' \times d_v} \\
  \left[\text{Attention}(Q, K, V)\right]_{i'k} &= \softmax_i \left( \frac{Q_{i'j} K_{ij}}{\sqrt{d_k}} \right) V_{ik}.
\end{align*}
Because $i'$ and $k$ appear on both sides, the equation must hold over all values of these indices. But because $j$ and $k$ occur twice on only the right-hand side, they are both summed over. We'd have to define exactly what the $i$ under softmax means ($i$ is bound inside the softmax and free outside it), and since softmax doesn't distribute over addition, we'd need to clarify that the summation over $j$ occurs inside the softmax.

Other than that, this is concise and unambiguous. But it doesn't really solve the main problem we set out to solve, which is that ordered axes force the author and reader to remember the purpose of each axis. The indices do act as symbolic names for axes (indeed, in \emph{abstract} index notation, they really are symbols, not variables), but they are temporary names; they could be totally different in the next equation. It would be up to the author to choose to use consistent names, and to do so correctly.

A second issue is that because it depends on repetition of indices to work, index notation can be a little bit more verbose than our notation, particularly for reductions and contractions:
\begin{align*}
  C &= \max_i A_i & C &=\nfun{ax}{max} A \\
  C &= A_i B_i & C &= A \ndot{ax} B.
\end{align*}

Finally, index notation requires us to write out all indices explicitly. So if we wanted to extend attention to multiple heads and minibatches, we would write:
\begin{gather*}
  \text{Attention} \colon \reals^{B \times H \times n' \times d_k} \times \reals^{B \times H \times n \times d_k} \times \reals^{B \times H \times n \times d_v} \rightarrow \reals^{B \times H \times n' \times d_v} \\
  \left[\text{Attention}(Q, K, V)\right]_{bhi'k} = \softmax_i \left( \frac{Q_{bhi'j} K_{bhij}}{\sqrt{d_k}} \right) V_{bhik}.
\end{gather*}
We could adopt a convention that extends a function on tensors to tensors that have extra axes to the \emph{left}, but such conventions tend to lead to messy reordering and squeezing/unsqueezing of axes. Named axes make this unnecessary.



\section{Conclusions}
Named tensor notation is a system of formal notation for representing operations between tensors in a non-ambiguous way while remaining intuitive for practitioners. The system is motivated by challenges that arise from taking notation designed for applied linear algebra and using it for representing neural networks, as demonstrated through examples of canonical deep-learning components such as attention and layer normalization. However, named tensors are not limited to specifying neural networks. We have also explained how to integrate our notation with \citet{magnus+neudecker:1985}'s method of differentials for matrix calculus. While there are other conventions that such as index notation that have some usage in the machine learning community, these conventions either lack the conciseness of named tensors or are not well-suited to non-linear operations. For these reasons, we encourage members of the machine learning community to try out named tensor notation for teaching, research, and software documentation.

\section*{Acknowledgements}

We would like to thank Ekin Aky\"{u}rek, Justin Bayer, Tongfei Chen, Chu-Cheng Lin, Colin McDonald, Adam Poliak, Matt Post, Chung-chieh Shan, Nishant Sinha, and Yee Whye Teh for their input to this document (or the ideas in it). We also thank the anonymous TMLR reviewers for their feedback, which substantially improved the quality of the paper, especially \cref{sec:calculus}.

This material is based upon work supported by the National Science Foundation under Grants No.~CCF-2019291 and~DMS-213415, as well as Simons Investigator Fellowship, DARPA grant W911NF2010021, and DOE grant DE-SC0022199. 
Any opinions, findings, and conclusions or recommendations expressed in this material are those of the authors and do not necessarily reflect the views of the funding agencies.

\iffalse % hack to make this heading appear only in pandoc
\section*{References}
\fi
\bibliographystyle{tmlr}
\bibliography{references}

\appendix

\section{Extended Examples}
\label{sec:examples_long}
\subsection{Transformer}
\label{sec:transformer}

We define a Transformer used autoregressively as a language model.
The input is a sequence of one-hot vectors, from which we compute word embeddings and positional encodings:
\begin{align*}
  I &\in \{0, 1\}^{\seq \times \vocab} & \nsum{\vocab} I &= 1 \\
  W &= (E \ndot{\vocab} I)\sqrt{|\layer|} & E &\in \reals^{\vocab \times \layer} \\
  P &\in \reals^{\seq \times \layer} \\
  P_{\seq(p), \layer(i)} &= \begin{cases}
    \sin((p-1) / 10000^{(i-1) / |\layer|}) & \text{$i$ odd} \\ 
    \cos((p-1) / 10000^{(i-2) / |\layer|}) & \text{$i$ even.}
  \end{cases}
\end{align*}

Then we use $L$ layers of self-attention and feed-forward neural networks:
\begin{align*} 
X^0 &= W+P \\
T^1 &= \text{LayerNorm}^1(\text{SelfAtt}^1(X^0) + X^0)\\
X^1 &= \text{LayerNorm}^{1'}(\text{FFN}^1(T^1) + T^1)\\
&\vdotswithin{=} \\
T^{L} &= \text{LayerNorm}^L(\text{SelfAtt}^L(X^{L-1}) + X^{L-1})\\
X^{L} &= \text{LayerNorm}^{L'}(\text{FFN}^L(T^L) + T^L)\\
O &= \nfun{\vocab}{softmax}(E \ndot{\layer} X^L)
\end{align*}
where $\text{LayerNorm}$, $\text{SelfAtt}$ and $\text{FFN}$ are defined below.

Layer normalization ($l = 1, 1', \ldots, L, L'$):
\begin{align*}
  \text{LayerNorm}^l \colon \reals^{\layer} &\rightarrow \reals^{\layer} \\
  \text{LayerNorm}^l(X) &= \nfun{\layer}{standardize}(X) \odot \gamma^l + \beta^l \\
  \beta^l, \gamma^l &\in \reals^{\layer}
\end{align*}

We defined attention in \S\ref{sec:attention}; the Transformer uses multi-head self-attention, in which queries, keys, and values are all computed from the same sequence.
\begin{align*}
  \text{SelfAtt}^l \colon \reals^{\seq \times \layer} &\rightarrow \reals^{\seq \times \layer} \\
  \text{SelfAtt}^l(X) &= Y
\end{align*}
where
\begin{align*}
  |\seq| &= |\seq'| \\
  |\key| = |\val| &= |\layer|/|\heads| \\
  Q &= W^{l,Q} \ndot{\layer} X_{\seq\rightarrow\seq'} & W^{l,Q} &\in \reals^{\heads \times \layer \times \key} \\
  K &= W^{l,K} \ndot{\layer} X & W^{l,K} &\in \reals^{\heads \times \layer \times \key} \\
  V &= W^{l,V} \ndot{\layer} X & W^{l,V} &\in \reals^{\heads \times \layer \times \val} \\
  M & \in \reals^{\seq \times \seq'} \\
  M_{\seq(i), \seq'(j)} &= \begin{cases}
    0 & i \leq j\\
    -\infty & \text{otherwise}
  \end{cases} \\
  Y &= W^{l,O} \ndot{\heads \\ \val} \text{Attention}(Q, K, V, M)_{\seq'\rightarrow\seq} & W^{l,O} &\in \reals^{\heads \times \val \times \layer}
\end{align*}

Feedforward neural networks:
\begin{align*}
  \text{FFN}^l \colon \reals^{\layer} &\rightarrow \reals^{\layer} \\
  \text{FFN}^l(X) &= X^2
\end{align*}
where
\begin{align*}
  X^1 &= \text{relu}(W^{l,1} \ndot{\layer} X + b^{l,1}) & W^{l,1} &\in \reals^{\hidden \times \layer} & b^{l,1} &\in \reals^{\hidden} \\
  X^2 &= W^{l,2} \ndot{\hidden} X^1 + b^{l,2} & W^{l,2} &\in \reals^{\layer \times \hidden} & b^{l,2} &\in \reals^{\hidden}.
\end{align*}

\subsection{LeNet}

\begin{align*}
X^0 &\in \reals^{\batch \times \chans[c_0] \times \height \times \width} \\
T^1 &= \text{relu}(\text{Conv}^1(X^0)) \\
X^1 &= \text{MaxPool}^1(T^1) \\
T^2 &= \text{relu}(\text{Conv}^2(X^1)) \\
X^2 &= \text{MaxPool}^2(T^2)_{(\height,\width,\chans)\rightarrow\layer} \\
X^3 &= \text{relu}(W^3 \ndot{\layer} X^2 + b^3) & W^3 &\in \reals^{\hidden \times \layer} & b^3 &\in \reals^{\hidden} \\
O &= \nfun{\classes}{softmax} (W^4 \ndot{\hidden} X^3 + b^4) & W^4 &\in \reals^{\classes \times \hidden} & b^4 &\in \reals^{\classes}
\end{align*}
As an alternative to the flattening operation in the equation for $X^2$, we could have written
\begin{align*}
X^2 &= \text{MaxPool}^2(T^2) \\
X^3 &= \text{relu}(W^3 \ndot{\height \\ \width \\ \chans} X^2 + b^3) & W^3 &\in \reals^{\hidden \times \height \times \width \times \chans}.
\end{align*}

The convolution and pooling operations are defined as follows:
\begin{align*}
\text{Conv}^l(X) &= \text{Conv2d}(X; W^l, b^l)_{\chans'\rightarrow\chans}
\end{align*}
where
\begin{align*}
W^l & \in \reals^{\chans'[c_l] \times \chans[c_{l-1}] \times \kh[kh_l] \times \kw[kw_l]} \\
b^l &\in \reals^{\chans'[c_l]}
\end{align*}
and
\begin{align*}
\text{MaxPool}^l(X) &= \text{MaxPool2d}_{ph^l,ph^l}(X).
\end{align*}


\section{Differentiation: Formal Definitions}
\label{sec:calculus_formal}
The following definition and theorem come directly from the paper by \citet{magnus+neudecker:1985}, but generalized to named tensors.

For any $X \in \reals^\mathcal{S}$, we write $\|X\| = \nfun{\mathcal{S}}{norm} X$.

\begin{definition}
  Let $f \colon S \rightarrow \reals^\mathcal{T}$ where $S \subseteq \reals^\mathcal{S}$.
  Let $A$ be an interior point of $\reals^\mathcal{S}$, that is, for some $r > 0$, $B(A;r) = \{X \mid \|X-A\| < r\} \subseteq S$.
  If there is a tensor $D(A) \in \reals^{\mathcal{S\inax} \cup \mathcal{T}}$ and $R(A,H) \in \reals^{\mathcal{T}}$ such that
  \begin{equation*}
    f(A + H) = f(A) + D(A) \ndot{\mathcal{S\inax}} H_{\mathcal{S} \rightarrow \mathcal{S\inax}} + R(A,H)
  \end{equation*}
  for all $H \in \reals^\mathcal{S}$ with $\|H\| < r$, and
  \begin{equation*}
    \lim_{H \rightarrow \mathbf{0}} \frac{R(A,H)}{\|H\|} = \mathbf{0},
  \end{equation*}
  then $f$ is said to be \emph{differentiable} at $A$; the tensor
  \begin{equation*}
    \partial f(A; H) = D(A) \ndot{\mathcal{S\inax}} H_{\mathcal{S} \rightarrow \mathcal{S\inax}}
  \end{equation*}
  is then called the \emph{(first) differential of $f$ at $A$ with increment $H$}.
\end{definition}

\citeauthor{magnus+neudecker:1985} give their (first) identification theorem twice, once for vector-to-vector functions and once for matrix-to-matrix functions (but omitting vector-to-matrix and matrix-to-vector functions). Here, we only need one version, which works for functions from tensors to tensors of any shape.
\begin{theorem} \label{thm:identification}
  Let $f \colon S \rightarrow \reals^\mathcal{T}$, where $S \subseteq \reals^\mathcal{S}$, be differentiable at $A \in S$. Let $D(X) \in \reals^{\mathcal{S}\inax \cup \mathcal{T}}$. Then
\begin{align*}
\text{for all $H$,} \ \partial f(A; H) &= D(X) \ndot{\mathcal{S}\inax} H_{\mathcal{S}\rightarrow\mathcal{S}\inax} \qquad \text{iff} \qquad \left.\ddx{f(X)}\right|_{X=A} = D(X).
\end{align*}
\end{theorem}



\section{\LaTeX{} Macros}

Many of the \LaTeX{} macros used in this document are available in the style file \verb|namedtensor.sty|, available at \url{https://namedtensor.github.io/namedtensor.sty} or on CTAN. To use it, put
\begin{quote}
\begin{verbatim}
\usepackage{namedtensor}
\end{verbatim}
\end{quote}
in the preamble of your \LaTeX{} source file (after \verb|\documentclass{article}| but before \verb|\begin{document}|).

We write axis names in sans-serif font. To make this easier, \verb|\name{ax}| prints an axis name (like this: \name{ax}), and \verb|\ndef{\ax}{ax}| defines a macro \verb|\ax| that does the same.

\begin{itemize}
\item Binary operators
  \begin{itemize}
  \item Use \verb|A \ndot{\ax} B| for contraction: $A \ndot{\ax} B$. You can use \verb|\\| to stack up several names.
  \item In general, you can use \verb|\nbin| to make a new binary operator with a name under it: \verb|A \nbin{\ax}{\star} B| gives you $A \nbin{\ax}{\star} B$.
  \end{itemize}
\item Functions
  \begin{itemize}
  \item Use \verb|\nsum{\ax} A| for summation: $\nsum{\ax} A$.
  \item In general, you can use \verb|\nfun| to make a function with a name under it: \verb|\nfun{\ax}{qux} A| gives you $\nfun{\ax}{qux} A$.
  \end{itemize}
\end{itemize}


\end{document}
